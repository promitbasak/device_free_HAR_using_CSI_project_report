\chapter{INTRODUCTION}
 \label{Chapter 1}
 \lhead{ Chapter 1. \emph{Introduction}}
\label{intro}

Healthcare has always been a fundamental concern for human society, and it is especially needed for the sick and the elderly. The percentage of persons aged over 65 was 9\% in 2020, with the rate being as high as 28\% in a single country \cite{WB:2020}. For an elderly person or a sick patient, falling can be devastating, and immediate attention is required in such a scenario. The development of fall detection systems is essential for this reason. Although many fall detection systems exist today, most of them are based on wearable devices or computer vision, which have some critical drawbacks, caused by the fundamental nature of the systems. This project is aimed to provide a non-contact fall detection system that is free of those drawbacks and is yet as effective as the existing solutions. Furthermore, we use embedded devices as this makes the system highly modifiable, deployable in many different environments, and comparatively affordable. 


\section{Motivation}

Sensor-based activity detection requires the user to wear a device containing the sensors in many cases\cite{7841080}. Some methods use computer vision for the task, which requires a camera to collect video or image data\cite{visionFall}. Although these methods can detect a fall incident quite accurately, some issues caused by these methods can make implementing them in real-world scenario a challenge. The issues faced while implementing solutions based on these methods are: 


\begin{itemize}
\item A wearable device can be perceived as uncomfortable, causing unwillingness to use them.
\item Remembering to wear a device every day can be an issue for elderly people.
\item A wearable device is more prone to wear and tear than a stationary device.
\item A camera-based solution can be both computationally and monetarily expensive.
\item A camera-based solution cannot provide reliability in adverse lighting conditions.
\end{itemize}

Comfort is of utmost importance for the elderly and the sick. A user might be understandably unwilling to wear a device if it is uncomfortable for them. Even if a system is perfectly capable of performing its assigned task, implementing it is challenging, if not impossible when the users are not willing to cooperate. Also to have a widespread application of a system, cost and durability have to be considered, especially in developing countries. So, in this project, we aim to create a non-contact fall detection system using wifi channel state information that is more comfortable, more reliable, and less costly.


\section{Objectives}

 We aim to build a non-contact fall detection system for monitoring the elderly and the sick. The objectives we aim to achieve are as follows:

\begin{itemize}
\item Implementing a system that can detect if a person in its area of operation has fallen down
\item Ensuring comfort by eliminating the need for wearing any device
\item Creating a dataset of channel state information recorded during different activities including fall
\item Recognition of different human activities using Wi-Fi channel state information to facilitate future improvement opportunities
\item Implementing a system to detect if the area of operation is empty
\item Ensuring that the system is easy to deploy and affordable by using embedded devices
\end{itemize}
